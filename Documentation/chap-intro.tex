\chapter{Introduction}
\pagenumbering{arabic}%

\section{Purpose}

\sysname{} is an assembler \cite{Salomon:1992:AL:152201}, but it
differs from traditional assemblers in some crucial ways:

\begin{itemize}
\item It does not take a \emph{source file} as its input.  Instead,
  instructions and labels are represented as \emph{standard
    objects}.%
\footnote{Recall that ``a standard object is an instance of (a
  subclass of) the class named \texttt{standard-object}.  A standard
  object is what you obtain when you call \texttt{make-instance} of
  a class defined with \texttt{defclass}.}  This way, we avoid the
problem of having to define an input syntax for instructions.
\item It does not produce \emph{object code} as output.  Instead, it
  produces a \commonlisp{} vector of unsigned eight-bit bytes.
\item There must not be any unresolved references in the program
  submitted to \sysname{}.  All references to labels must have a
  corresponding label defined.
\end{itemize}

\sysname{} is mainly meant to server as a \emph{backend} for compilers
written in \commonlisp{}, but they can be compilers for any language.
With \sysname{}, the compiler does not need to turn instructions into
surface syntax or S-expressions, only to have the assembler parse that
output into some internal representation right away.  By avoiding this
pair of unparsing/parsing step, the speed of the compilation process
is slightly improved.  But the main reason for \sysname{} to avoid
surface syntax is that it can easily become ambiguous over time, as
more types of instructions and operands need to be expressed.  It is
much easier to extend a \clos{} class to handle new situations as the
arise.
